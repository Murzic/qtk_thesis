\section*{Conclusions}
\phantomsection


QTK was developed with the idea of simplifying and partially automatizing the managing of quizzes in the educational system. The author of the thesis had to create an application which would centralize the quizzes created by the teachers, by giving them an online quiz management system with additional features of generating quizzes based on different options and automatically processing the scanned papers of the quizzes so that they can be properly graded. All these features are tied together to an intuitive user interface that is welcomed by the plain design of the system. 

The purpose of the project was to help people in this educational domain. The target groups of it are lecturers(teachers) and students. They are given the chance of saving time when dealing with quizzes in their teaching time. But this is just one way of looking at it. Another benefit would be introducing quizzes more often into current curriculum. This can be achieved by demonstrating how easy it is to use quizzes within this application. The proof of that can be reached by giving QTK a try. Of course, people usually like being in their zone of comfort and they are reluctant to try something new, so the best option would be to introduce this system to the whole faculty or university. This would definitely direct teachers into using it and simplifying their teaching style. 

The project was implemented using Ruby on Rails, a web-application framework based on the MVC pattren. Different additional libraries(gems) were used so that not all the particularities of the system would be developed from scratch. These libraries were used to create QR codes, to decode them, to attach image files to model classes, to process images, to access images' pixel colors and to run jobs in the background. Of course, other default libraries were already included by default by the Rails framework, when the project was initiated. 

The implementation can be divided in three main phases:
\begin{enumerate}
  \item Creation of the quiz management system
  \item Developing the quiz generation part
  \item Developing the quiz's scan processing part
\end{enumerate}

The creation of the quiz management system was related to designing the database, creating the necessary user interface and the controllers which react to the users' requests. \textit{Bootstrap} was used as a front-end framework to make the user interface good-looking and neat. 

The main focus was placed on the last two phases, which are considered the most important and without which, the whole project wouldn't make sense. The second phase required the developer to add to the system the possibility to generate quizzes, by offering the user to select from a list of options, which would be used to model the specific type of quiz. The final result, would be a pdf file containing all the copies to be printed and handed to students. Each copy page would contain the questions and answers corresponding to the quiz, the student's and group's name, if the necessary option was provided, the QR code, encoding the copy id and the current page and the markers, which would guide the image processing into finding the coordinates of the check-boxes.

In the third phase, the image processing implementation, uses the previosly included elements of the generated pages to make the necessary computations to record the students' answers without having to check them by hand. This phase also includes the grading, in percentage, of the students' success. The image processing is comprised of the decoding of the QR code, which gives us the necessary information to access the questions from the database corresponding to the specific page from which the QR code was decoded. The markers are used as originating points from which the image coordinates will be accessed. Also they are used to compute the angle the text is rotated, due to the printing and scanning errors. This angle is used to adjust the pixels' coordinates which are going to be accessed. This make this adjustment precise, was the biggest challenge met by far. An initial attempt of using border lines instead of circles in the corners as markers, proved to be a failure, which made the time spent on this attempt almost wasted. It still gave the developer a base, from which the new approach could be taken. Another thing worth mentioning, is the use of the background jobs to do the image processing. This was decided, when it was noticed that the image processing is time consuming and the user's waiting of the servers response was too long. So, making this specific task to run asynchronously was pretty rewarding, by making it imperceptible by the user.

After completing these phases, the system system is fully operational and is performing as it was intended to. Still, there is a lot of room for improvement. Begining with the user interface, which was not polished till the end and ending with with the generation of quizzes process. The quiz generation process could be run in the background the same as the quiz processing job. Features as plotting graphs with different statistics, sharing quizzes between users, more flexible manipulation with the already generated quizzes can be implemented in the future. All these improvements are optional, and the system can be used as it is.




\clearpage