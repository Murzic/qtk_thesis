\section*{Conclusions}
\phantomsection

Conclusions
-what did you create? 
-how the model was designed?
- main focus was on ….? 
-stages of work...stage one...stage two...stage three
- challenges, how did you solve them?
- development of interface

QTK was developed with the idea of simplifying and partially automatizing the managing of quizzes in the educational system. The author of the thesis had to create an application which would centralize the quizzes created by the teachers, by giving them an online quiz management system with additional features of generating quizzes based on different options and automatically processing the scanned papers of the quizzes so that they can be properly graded. All these features are tied together to an intuitive user interface that is welcomed by the plain design of the system. 

The purpose of the project was to help people in this educational domain. The target groups of it are lecturers(teachers) and students. They are given the chance of saving time when dealing with quizzes in their teaching time. But this is just one way of looking at it. Another benefit would be introducing quizzes more often into current curriculum. This can be achieved by demonstrating how easy it is to use quizzes within this application. The proof of that can be reached by giving QTK a try. Of course, people usually like being in their zone of comfort and they are reluctant to try something new, so the best option would be to introduce this system to the whole faculty or university. This would definitely direct teachers into using it and simplifying their teaching style.

  % A very important thing to mention is the QTK output. What the program brings to the society, to those who are meant to use it in their professional activity? The purpose of the project was to serve to people. The QTK target groups are lecturers and students. Some may be in doubt about user "readiness" taking in consideration that in the given environment not entire academic staff is familiarized with such practices. It just cannot be successful without appropriate, well-supported and focused human intervention, good learning design or pedagogical input and the sensitive handling of the process over time by trained tutors. That is why all these were taken into account when designing the application. The workflow of the program corresponds to the possibilities and needs of the potential users, keeping a  user friendly interface and demanding a simple human-technology interaction.
  % When talking about e-Learning tools someone may think that it mandatory implies the presence of internet connection or/and mass use of expensive electronic devices. The QTK demonstrates that this concept is misunderstood. The offline quiz module may be used at different universities for mass exams. Hundreds of students can be easily examined at the same time without the need for expensive e-testing equipment. This is a strong argument that should encourage the higher education managers to apply QTK in the educational delivery environment. Another benefit the app is that it brings the education to a higher level (changing the pedagogical approach by diversification of quiz giving/taking/evaluating process). It is oriented towards students too, who get a personalized approach and a fair evaluation of knowledge. 
  % More than that, the implementation of the program and its efficient function and maintenance during the academic year implies little budget. This  is not trivial for universities when facing the decline of public funding, raising expenses, increasingly diverse student bodies and their changing need  and heightened demand for new and different services and programs.

-what else could be done in future? what features can be added?
  Even though the results of the thesis met the expectations, the program works properly and  executes accurately all the commands, there is always room for better.  …………..



\clearpage