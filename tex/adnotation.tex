\section*{Adnotation}
% Thesis Mathematical models and methods for improving the conversion efficiency
% of renewable energy based on aero-hydrodynamics effects, presented by Viorel Bostan for
% the competition of doctor habilitate degree in techincal sciences, was developed at the Technical
% University of Moldova , Chisinau, is written in Romanian and contains 433 pages, 145 figures, 8 tables,
% and 223 references. The thesis consists of introduction, six chapters, conclusions and appendices.
% Appendicies contain additional 57 figures and 49 tables.

Thesis {\bf Quiz ToolKit}(QTK), presented by Eugeniu Ungureanu as a Bachelor project, was developed at the Technical University of Moldova in Chi\c{s}in\u{a}u. It is written in English and contains 17 figures, 10 tables and 11 references. The thesis consists of list of figures, list of tables, list of listings, introduction, four chapters, conclusion and references list.

This thesis is dedicated to the study of asynchronous execution of certain tasks in a web based environment, image processing and document rendering. The purpose of this thesis was to elaborate a system where quizzes would be stored, managed, generated for certain groups of people to be assessed and in the end, those quizzes would be scanned, uploaded into the system to be automatically processed and evaluated.

The problems that can be met when dealing with quizzes are the student's possibility to cheat, given that a few versions only are provided, the teacher's wasted time in creating these quizzes, designing them specifically for every group to be assessed and the process of verifying the completed quizzes and marking them. These problems leads to the quiz type of assessment unpopular and it is rarely used. Quiz ToolKit project is an attempt to dissolve these issues and provides a fast and objective, ready to use instrument for giving the students the opportunity to measure their strengths, weaknesses and their evolution using quizzes as a representative type of assessment.  

The chapters which depict this report are: the domain analysis chapter, the system design chapter, the implementation chapter and the economic analysis chapter. The first chapters defines the problem, which the developed system is going to solve. Also, it contains general information about different technologies, explanation of why these technologies were chosen and their specific usage examples. The next chapter contains the UML diagrams used for the designing of the system. It has an ample list of uses cases' descriptions defined in the incipient stage of the project development. The third chapter describes the implementation of the system. Examples of code are given in listings, which have their main aspects outlined. The quiz generation and quiz processing phases are explained in this chapter and examples in form of figures or listings are given. The last chapter analyses the system from an financial point of view. The expected expenses of the system's implementation are calculated. This document is for readers with technical background, engineers, IT students, and programmers. 
% The five chapters which compose our report are: the problem and domain analysis chapter, the implementation chapter, the security chapter, the other considerations chapter, and the economic analysis chapter. The first chapter describes the problem that our system is designed to solve, as well as all the necessary technology we'll use in order to make the system functional. The next chapter describes in depth how the project is implemented. Code listings will be shown and described in depth. For every component in the structure, its implementation will be explained. The third chapter examines the security of the system. If it is left vulnerable, future clients could be risking their privacy. The weak points, as well as solutions to eliminate them or secure them properly will be researched. In the other considerations chapter we will explore how the system fairs from a usability point of view, as well as future features that can be added, in order to make it better and more versatile. Finally, we have the economic analysis chapter, in which we analyze the project from a financial point. All the expenses and expected income will be computed. This document is for readers with technical background, engineers, IT students, and programmers. 


