\section{Economic Analysis}

\subsection{Project Description}
The success of an effective “e-learning” process in education is a good management and economic efficiency in its implementation and maintenance. Nowadays the main prerogative in education in the developed countries is to provide high-quality studies, mainly using the “e-learning” tool. The lack of the financial support or defective management in developing countries affect their educational system, by using the old-school methods and tools of teaching, depriving both professors and students of innovative ways of learning.  That is why the main aim of the current project is to create an accessible and simple to use app that ease the process of creating/evaluating quizzes. Quizzes are frequently used by professors to assess the learning process throughout the entire academic year. Taken often, for almost every chapter, the whole processes starting from creating to verifying the individualized copies may take sometimes precious time and useless effort. For that it is meant that the “Quiz ToolKit” application will be used directly by the academic staff who will introduce once manually the data on the PC: questions/answers which automatically will be stored and the app will generate single copies of desired tests, managing their level of complexity in the form of a pdf file. Thereafter these will be printed in hard copies and given to students. Later the completed hard copies will be scanned and the collected data will be processed and evaluated automatically. 

Due to the changes we are experiencing daily, continuous development of different spheres of life; the higher education institution must deal with a more competitive world and market forces than before. They are facing challenges as decline of the public funding and raising expenses. Student bodies have higher expectations, needs and demands for services and programs (Eckel et al. 2005:4 and Douglas, 2005:6). Important to notice that such institutions, who were historically (only) concerned with teaching, research and service now have to redirect their policies and focus their attention in searching for entrepreneurialism and commercialization. In such order of ideas institutions are willing to apply new technologies in the delivery of education to reach new student markets which will lead to a higher rate of enrolment (Douglas, 2005:2). For that, according to Bates (2000:44-45), institutions have to:

\begin{itemize}
  \item define a vision for teaching and learning and where the technologies fit in;
  \item identify the new target groups using technologies;
  \item define priority target groups appropriate technologies;
  \item identify the already-existing technologies support and encourage them to provide support for the new once;
  \item identify areas of support outside the institution;
  \item ensure that innovation and the skilled use of technology for teaching is properly recognized and rewarded
  \item identify the role of and priorities for face to face versus technology sophisticated technology     based learning;
  \item decide on key areas of investment and resource allocation for technology based teaching.
\end{itemize}

To create and launch such kind of application, which will enhance the individual approach and sustain innovation it is necessary to increase the institutional desire to implement the “e-learning” process by suggesting an efficient management and making an economical analysis as well.

\subsection{SWOT Analysis}
SWOT analysis is the study of project’s strengths and weaknesses, opportunities and threats from the very different points of view, including the economical one. The primary aim is to predict how well a product would succeed on the market, in order to comply with revenue expectations and prepare for possible risks that may occur. An educational application can be judged in such a way as well. The analysis of this project is presented in table \ref{swot_analysis}.

By taking a look at SWOT analysis it can be concluded that the project has strong as well as      weak points, and even though institutions are usually resource-constrained, the use of accurate marketing techniques will help to present and sell the project in a better way. Opportunities and threats present a sketchy prediction of the direction of app development on the one hand, and on another the risks that may occur.

\begin{table}[ht!]
\centering
\caption{SWOT Analysis}
{
\renewcommand{\arraystretch}{2}
\begin{tabular}{ p{8cm} p{8cm} }

  {\bf Strong  Points } & {\bf Weak Points }\\
  -- Built for learning; & 
  -- Need to consider scanning quality; \\
  -- Designed to support teaching; & 
  -- Need to consider the human factors when scanning (correct page position) \\
  -- Easy to use; & 
  -- User language limited (English only); \\
  -- Automatically generates quizzes and different options for generation; & 
  -- Authentication (only basic, no Google, Facebook, LinkedIn, Github, or Windows authentication). \\
  -- Automatically processes quizzes; \\
  -- Manages quizzes; \\
  -- Quizzes can be generated for specific student groups; \\
  -- Free with no licensing fees; \\
  -- No need of expensive e-testing equipment. \\

  {\bf Opportunities }& {\bf Threats } \\
  -- Possibility to use anytime, anywhere, on any device; &
  -- Existence of a similar application. \\
  -- Highly flexible and fully customizable possibility of format and interface; \\
  -- All in one learning platform (possibility to configure features and integrate everything needed for the course); \\
  -- Possibility of creating ranks and charts. \\

\end{tabular}
}
\label{swot_analysis}
\end{table}

\subsection{Project time schedule}
Time management is one of first and important processes to be completed when designing an application. It is necessary for organizing in order to meet the deadlines. The best guide in planning is past experience, and not having one may result into unexpected difficulties and delayed results. Nevertheless, objectives can be met successfully. For that project’s objectives have to be prioritized and given a sufficient amount of time needed to meet the deadline following some general known rules of time managing. Presuming the time needed to create the "Quiz ToolKit", some room in the schedule was left for the unexpected. Some time was left to account for unforeseen issues, so that they didn't negatively impact other tasks.
\subsubsection{Objectives}
The main objective of the “Quiz ToolKit” is to provide a powerful set of learning tools for the educational institutions. This implies that the platform will work in two stages: first- preparation for the quiz, and second- evaluation of the results. The objectives for the first one are to successfully store the data: questions/answers needed for the test and create individualized quizzes automatically. After the quizzes will be printed out on hard copies and given to students to fill in and professors will collect them back begins the second phase. Objectives set for the second phase include a successful scan of the hard copies and processing of the results, including grading. 
The economical main objective of the application is seeking for external investors, being a free platform with no licensing fees, attracting them with its utility and personalized learning environment. 

\subsubsection{Schedule}
There are five basic phases of how the project was done. These are:
\begin{enumerate}
  \item Project conception and initiation;
  \item Project planning;
  \item Project implementation;
  \item Project validation; 
  \item Launch. 
\end{enumerate}
These are subdivided in smaller ones and working on the project was following the cycle ”iterative model” because at the beginning there was no clear idea from what to start and what tasks should be made next. Development was done by implementing one part of the software, which led to new ideas of how to continue the work. Further requirements were determined and again these ones showed to path which had to be done. 

People involved in the development of the application are:
\begin{itemize}
  \item Project Manager (PM) - person who coordinated and supervised the developer; 
  \item Software Developer (SD) - person who designed and developed the application. 
\end{itemize}
The formula (\ref{4.1}) represents the total duration of the project:
\begin{equation}
D = D_{S} - D_{E} + R  \label{4.1}
\end{equation}

Where $D$ is the duration, $D_{S}$ is the start date, $D_{E}$ is the end day and $R$ is the reserve time. Using the above information and formula the initial schedule of the project is presented in Table \ref{project_schedule}.  Table \ref{project_schedule} represents actions that need to be done to evaluate the project, time required for each action, workers and resources needed. The total time to complete the project is {\bf93 days}.

\begin{itemize}
  \item PM (project manager): 9 days;
  \item SD (software developer): 93 days.
\end{itemize}

\begin{table}[ht!]
\centering
\caption{Project Schedule}
{
\renewcommand{\arraystretch}{1.3}
\begin{tabular}{ cp{5cm}cc  p{6.5cm} }
    & {Activity name} & \pbox{2cm}{Duration (days)} & Workers & {Resources Used} \\
    \hline
  1 & Analysis of tasks & 2 & PM, SD & Internet, PC, office, inventory (paper, pen, etc) \\
  2 & Requirements setting & 2 & PM, SD & Internet, PC, office \\
  3 & Study available technologies & 10 & SD & Internet, PC, books, office \\
  4 & Interface design & 5 & PM, SD & Internet, PC, office \\
  5 & System design & 10 & SD & PC, office, UML tool, Internet \\
  6 & System implementation & 30 & SD & PC, office, Internet \\
  7 & Testing and adjustments & 15 & SD & PC, office, printer, scanner, Internet \\
  8 & Project documentation & 15 & SD & PC, office, Internet \\
  9 & Project presentation preparations & 4 & SD & PC, office, Internet \\ \hline
    & Total & 93 
\end{tabular}
}
\label{project_schedule}
\end{table}

\subsection{Economical proof}

Expenses should be computed in order to evaluate the project from the economical point of view. Following groups of expenses are divided into: tangible, intangible, salary and indirect expenses. Because this is a non-commercial project there are neither profit estimations nor financial results. Nevertheless in this section all the expenses, including the salary for workers, wear and depreciation of materials is computed. The budget will include the money necessary to buy all tangible/intangible assets, indirect expenses, salary. 

\subsubsection{Tangible and intangible expenses}
Tangible and intangible expenses have been computed to find out the budget needed for the project. The complete list of material assets are presented in the Table \ref{longterm_asset_expenses} and \ref{direct_expenses}. For this project there are no intangible assets, because all the software, applications and programs are either free of charge or offer a trial version. 

\begin{table}[ht!]
\centering
\caption{Long Term Asset Expenses}
{
\renewcommand{\arraystretch}{1.25}
\begin{tabular}{ llll }
\hline
  Name & Unit Price (MDL) & Quantity & Sum (MDL) \\ \hline
  PC & 10000 & 1 & 10000 \\
  Printer & 7000 & 1 & 7000 \\
  Fast Scanner & 36500 & 1 & 36500 \\
  Total & & & 53500 \\
\hline
\end{tabular}
}
\label{longterm_asset_expenses}
\end{table}

\begin{table}[ht!]
\centering
\caption{Direct Expenses}
{
\renewcommand{\arraystretch}{1.25}
\begin{tabular}{ llll }
\hline
  Name & Unit Price (MDL) & Quantity & Sum (MDL) \\ \hline
  Copybook & 30 & 1 & 30 \\
  Printing paper & 89 & 1 & 89 \\
  Pen & 5 & 1 & 5 \\
  Printing & 0.5 & 200 & 100 \\
  Scanning & 1 & 200 & 200 \\
  Total & & & 424 \\
\hline
\end{tabular}
}
\label{direct_expenses}
\end{table}


\subsubsection{Salary expenses}
In this chapter the expenses needed to remunerate the labor personnel will be counted. Certain considerations will be taken into account as well, such as current percentage for the various funds that need to be paid. Medical insurance and social funds will be computed as well. The salaried are presented in Table \ref{salary_expenses}. Having this it is clear to compute how much has to be paid for the social service fund, medical insurance fund and the total work expenses that will be obtained by summing those. 


\begin{table}[ht!]
\centering
\caption{Salary Expenses}
{
\renewcommand{\arraystretch}{1.25}
\begin{tabular}{ lllll }
\hline
  Position &  Nr. of employees & Amount of work(h) & Sal/unit (MDL/h) & FSB (MDL) \\ \hline
  Project Manager & 1 & 72 & 120 & 8640 \\
  Software Developer & 1 & 744 & 90 & 66960 \\
  Total & & & & 75600 \\
\hline
\end{tabular}
}
\label{salary_expenses}
\end{table}


$F_{re}$ represents “Fondul de Retribuire a Muncii”, and is equal to: 
\begin{equation}
  F_{re} = 8640 + 66960 = 75600 (MDL) \label{4.2}
\end{equation}

The social service expenses will be equal to:
\begin{equation}
  FS = F_{re} * T_{fs} = 75600 * 0.23 = 17388 (MDL) \label{4.3}
\end{equation}
where $T_{fs}$ is the contribution quota for the state mandatory social insurance, approved annually by the “Law of Budget” (in 2015 – 23\%). Now the medical insurance is computed as:
\begin{equation}
  MI = F_{re} * T_{mi} = 75600 * 0.045 = 3402 (MDL) \label{4.4}
\end{equation}
where $MI$ in the medical insurance and $T_{mi}$ is the medical insurance quota approved each ear by the “Law of Budget” for the state medical insurance (in 2015 – 4,5\%).

The total work expense fund can be computed as follows:
\begin{equation}
  WEF = F_{re} + FS + MI = 75600 + 17388 + 3402 = 96390 (MDL) \label{4.5}
\end{equation}
where $WEF$ is the work expense fund.
